\documentclass[12pt]{article}

\usepackage{setspace}

\usepackage{amsmath, amsfonts, amssymb, graphicx, color, fancyhdr, lipsum, scalerel, stackengine, mathrsfs, tikz-cd, mdframed, enumitem, framed, adjustbox, bbm, upgreek, xcolor, hyperref}
\usepackage[framed,thmmarks]{ntheorem}
\usepackage[style=alphabetic]{biblatex}
%Set the bibliography file
\bibliography{sources}

%Replacement for the old geometry package
\usepackage{fullpage}

%Input my definitions
\input{./mydefs.tex}

%Shade definitions
\theoremindent0cm
\theoremheaderfont{\normalfont\bfseries} 
\def\theoremframecommand{\colorbox[rgb]{0.9,1,.8}}
\newshadedtheorem{defn}[thm]{Definition}

%%%%%%%%%%%%%%%%%%%%%%%%%%%%%%%%%%%%%%%%%%%%%%%%%%%%%%%%%%%%%%%%%%%%%%
%%%%%%%%%%%%%%%%%%%%%%% Customize Below %%%%%%%%%%%%%%%%%%%%%%%%%%%%%%
%%%%%%%%%%%%%%%%%%%%%%%%%%%%%%%%%%%%%%%%%%%%%%%%%%%%%%%%%%%%%%%%%%%%%%

%header stuff
\setlength{\headsep}{24pt}  % space between header and text
\pagestyle{fancy}     % set pagestyle for document
\lhead{Notes on the Grassmannian} % put text in header (left side)
\rhead{Nico Courts} % put text in header (right side)
\cfoot{\itshape p. \thepage}
\setlength{\headheight}{15pt}
\allowdisplaybreaks

% Document-Specific Macros
\DeclareMathOperator{\Gr}{Gr}
\DeclareMathOperator{\GL}{GL}
\let\k\relax
\newcommand{\k}{\mathbbm{k}}

\begin{document}
%make the title page
\title{The Grassmannian as a Quotient of $\GL_n(\k)$ \vspace{-1ex}}
\author{Nico Courts}
\date{Summer 2019}
\maketitle

\renewcommand{\abstractname}{Introduction}
\begin{abstract}
	These notes are my summary of the realization of the Grassmanian $\Gr(n,k)$ as a quotient of 
	the Lie group $\GL_n(\k)$. In particular the focus will be on $\k=\bbR$ or $\bbC$.
\end{abstract}

\section*{Vista: Where We're Headed}
The idea that $\GL_n$ acts on vector subspaces of a Euclidean space should be unsurprising and natural, but 
the upshot to considering this viewpoint is that is allows us to consider this object from the perspective of smooth 
manifold theory. This gives us some great machinery to grasp onto to prove some nice properties 
about $\Gr(n,k)$.

\section{The Group Action}
Let $\k=\bbR$ or $\bbC$ and let $V=\k^n$. Let $G=\GL(V)$, considered as a Lie group. The first thing to notice is that there is a natural action 
of $G$ on the set of $k$-dimensional subspaces of $V$ in the following way:
\section{Smooth Manifold Structure}


%%%%%%%%%%%%%%%%%%%%%%%%%%%%%%%%%%%%%%
%%%%%%%%%%  Bibliography %%%%%%%%%%%%%
%%%%%%%%%%%%%%%%%%%%%%%%%%%%%%%%%%%%%%
\medskip

\printbibliography

\end{document}