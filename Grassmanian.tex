\documentclass[12pt]{article}

\usepackage{setspace}

\usepackage{amsmath, amsfonts, amssymb, graphicx, color, fancyhdr, mathrsfs, tikz-cd, mdframed, enumitem, framed, adjustbox, bbm, upgreek, xcolor, hyperref}
\usepackage[framed,thmmarks]{ntheorem}
\usepackage[style=alphabetic]{biblatex}
%Set the bibliography file
\bibliography{sources}

%Replacement for the old geometry package
\usepackage{fullpage}

%Input my definitions
%set up theorem/definition/etc envs
%Problems will be created using their own counter and style
\theoremstyle{break}
\theoreminframepreskip{0pt}
\theoreminframepostskip{0pt}
\newframedtheorem{prob}{Problem}[section]

%solution template
\theoremstyle{nonumberbreak}
\theoremindent0.5cm
\theorembodyfont{\upshape}
\theoremseparator{:}
\theoremsymbol{\ensuremath\spadesuit}
\newtheorem{sol}{Solution}

%Theorems
\definecolor{thmcol}{RGB}{120,100,50}
\theoremstyle{changebreak}
\theoremseparator{}
\theoremsymbol{}
\theoremindent0.5cm
\theoremheaderfont{\color{thmcol}\bfseries} 
\newtheorem{thm}{Theorem}[subsection]

%Lemmas and Corollaries
\theoremheaderfont{\bfseries}
\newtheorem{lem}[thm]{Lemma}
\newtheorem{cor}[thm]{Corollary}
\newtheorem{prop}[thm]{Proposition}

%Create a new env that references a theorem and creates a 'primed' version
%Note this can be used recursively to get double, triple, etc primes
\newenvironment{thm-prime}[1]
  {\renewcommand{\thethm}{\ref{#1}$'$}%
   \addtocounter{thm}{-1}%
   \begin{thm}}
  {\end{thm}}

\setlength\fboxsep{15pt}

%Example
\theoremstyle{break}
\def\theoremframecommand{\colorbox[rgb]{0.9,0.9,0.9}}
\newshadedtheorem{ex}{Example}[section]

%Man, that's really good! Let's use the same thing for definitons.
\newenvironment{def-prime}[1]
  {\renewcommand{\thethm}{\ref{#1}$'$}%
   \addtocounter{thm}{-1}%
   \begin{def}}
  {\end{def}}

%proofs
\theoremstyle{nonumberbreak}
\theoremindent0.5cm
\theoremheaderfont{\sc}
\theoremseparator{}
\theoremsymbol{\ensuremath\spadesuit}
\newtheorem{prf}{Proof}

\theoremstyle{nonumberplain}
\theoremseparator{:}
\theoremsymbol{}
\newtheorem{conj}{Conjecture}

%remarks
\theoremstyle{change}
\theoremindent0.5cm
\theoremheaderfont{\sc}
\theoremseparator{:}
\theoremsymbol{}
\newtheorem{rmk}[thm]{Remark}

%Put page breaks before each part
\let\oldpart\part%
\renewcommand{\part}{\clearpage\oldpart}%

% Blackboard letters
\newcommand*{\bbA}{\mathbb{A}}
\newcommand*{\bbB}{\mathbb{B}}
\newcommand*{\bbC}{\mathbb{C}}
\newcommand*{\bbD}{\mathbb{D}}
\newcommand*{\bbE}{\mathbb{E}}
\newcommand*{\bbF}{\mathbb{F}}
\newcommand*{\bbG}{\mathbb{G}}
\newcommand*{\bbH}{\mathbb{H}}
\newcommand*{\bbI}{\mathbb{I}}
\newcommand*{\bbJ}{\mathbb{J}}
\newcommand*{\bbK}{\mathbb{K}}
\newcommand*{\bbL}{\mathbb{L}}
\newcommand*{\bbM}{\mathbb{M}}
\newcommand*{\bbN}{\mathbb{N}}
\newcommand*{\bbO}{\mathbb{O}}
\newcommand*{\bbP}{\mathbb{P}}
\newcommand*{\bbQ}{\mathbb{Q}}
\newcommand*{\bbR}{\mathbb{R}}
\newcommand*{\bbS}{\mathbb{S}}
\newcommand*{\bbT}{\mathbb{T}}
\newcommand*{\bbU}{\mathbb{U}}
\newcommand*{\bbV}{\mathbb{V}}
\newcommand*{\bbW}{\mathbb{W}}
\newcommand*{\bbX}{\mathbb{X}}
\newcommand*{\bbY}{\mathbb{Y}}
\newcommand*{\bbZ}{\mathbb{Z}}
%Fraktur letters
\newcommand*{\frakA}{\mathfrak{A}}
\newcommand*{\frakB}{\mathfrak{B}}
\newcommand*{\frakC}{\mathfrak{C}}
\newcommand*{\frakD}{\mathfrak{D}}
\newcommand*{\frakE}{\mathfrak{E}}
\newcommand*{\frakF}{\mathfrak{F}}
\newcommand*{\frakG}{\mathfrak{G}}
\newcommand*{\frakH}{\mathfrak{H}}
\newcommand*{\frakI}{\mathfrak{I}}
\newcommand*{\frakJ}{\mathfrak{J}}
\newcommand*{\frakK}{\mathfrak{K}}
\newcommand*{\frakL}{\mathfrak{L}}
\newcommand*{\frakM}{\mathfrak{M}}
\newcommand*{\frakN}{\mathfrak{N}}
\newcommand*{\frakO}{\mathfrak{O}}
\newcommand*{\frakP}{\mathfrak{P}}
\newcommand*{\frakQ}{\mathfrak{Q}}
\newcommand*{\frakR}{\mathfrak{R}}
\newcommand*{\frakS}{\mathfrak{S}}
\newcommand*{\frakT}{\mathfrak{T}}
\newcommand*{\frakU}{\mathfrak{U}}
\newcommand*{\frakV}{\mathfrak{V}}
\newcommand*{\frakW}{\mathfrak{W}}
\newcommand*{\frakX}{\mathfrak{X}}
\newcommand*{\frakY}{\mathfrak{Y}}
\newcommand*{\frakZ}{\mathfrak{Z}}
\newcommand*{\fraka}{\mathfrak{a}}
\newcommand*{\frakb}{\mathfrak{b}}
\newcommand*{\frakc}{\mathfrak{c}}
\newcommand*{\frakd}{\mathfrak{d}}
\newcommand*{\frake}{\mathfrak{e}}
\newcommand*{\frakf}{\mathfrak{f}}
\newcommand*{\frakg}{\mathfrak{g}}
\newcommand*{\frakh}{\mathfrak{h}}
\newcommand*{\fraki}{\mathfrak{i}}
\newcommand*{\frakj}{\mathfrak{j}}
\newcommand*{\frakk}{\mathfrak{k}}
\newcommand*{\frakl}{\mathfrak{l}}
\newcommand*{\frakm}{\mathfrak{m}}
\newcommand*{\frakn}{\mathfrak{n}}
\newcommand*{\frako}{\mathfrak{o}}
\newcommand*{\frakp}{\mathfrak{p}}
\newcommand*{\frakq}{\mathfrak{q}}
\newcommand*{\frakr}{\mathfrak{r}}
\newcommand*{\fraks}{\mathfrak{s}}
\newcommand*{\frakt}{\mathfrak{t}}
\newcommand*{\fraku}{\mathfrak{u}}
\newcommand*{\frakv}{\mathfrak{v}}
\newcommand*{\frakw}{\mathfrak{w}}
\newcommand*{\frakx}{\mathfrak{x}}
\newcommand*{\fraky}{\mathfrak{y}}
\newcommand*{\frakz}{\mathfrak{z}}
% Caligraphic letters
\newcommand*{\calA}{\mathcal{A}}
\newcommand*{\calB}{\mathcal{B}}
\newcommand*{\calC}{\mathcal{C}}
\newcommand*{\calD}{\mathcal{D}}
\newcommand*{\calE}{\mathcal{E}}
\newcommand*{\calF}{\mathcal{F}}
\newcommand*{\calG}{\mathcal{G}}
\newcommand*{\calH}{\mathcal{H}}
\newcommand*{\calI}{\mathcal{I}}
\newcommand*{\calJ}{\mathcal{J}}
\newcommand*{\calK}{\mathcal{K}}
\newcommand*{\calL}{\mathcal{L}}
\newcommand*{\calM}{\mathcal{M}}
\newcommand*{\calN}{\mathcal{N}}
\newcommand*{\calO}{\mathcal{O}}
\newcommand*{\calP}{\mathcal{P}}
\newcommand*{\calQ}{\mathcal{Q}}
\newcommand*{\calR}{\mathcal{R}}
\newcommand*{\calS}{\mathcal{S}}
\newcommand*{\calT}{\mathcal{T}}
\newcommand*{\calU}{\mathcal{U}}
\newcommand*{\calV}{\mathcal{V}}
\newcommand*{\calW}{\mathcal{W}}
\newcommand*{\calX}{\mathcal{X}}
\newcommand*{\calY}{\mathcal{Y}}
\newcommand*{\calZ}{\mathcal{Z}}
% Script Letters
\newcommand*{\scrA}{\mathscr{A}}
\newcommand*{\scrB}{\mathscr{B}}
\newcommand*{\scrC}{\mathscr{C}}
\newcommand*{\scrD}{\mathscr{D}}
\newcommand*{\scrE}{\mathscr{E}}
\newcommand*{\scrF}{\mathscr{F}}
\newcommand*{\scrG}{\mathscr{G}}
\newcommand*{\scrH}{\mathscr{H}}
\newcommand*{\scrI}{\mathscr{I}}
\newcommand*{\scrJ}{\mathscr{J}}
\newcommand*{\scrK}{\mathscr{K}}
\newcommand*{\scrL}{\mathscr{L}}
\newcommand*{\scrM}{\mathscr{M}}
\newcommand*{\scrN}{\mathscr{N}}
\newcommand*{\scrO}{\mathscr{O}}
\newcommand*{\scrP}{\mathscr{P}}
\newcommand*{\scrQ}{\mathscr{Q}}
\newcommand*{\scrR}{\mathscr{R}}
\newcommand*{\scrS}{\mathscr{S}}
\newcommand*{\scrT}{\mathscr{T}}
\newcommand*{\scrU}{\mathscr{U}}
\newcommand*{\scrV}{\mathscr{V}}
\newcommand*{\scrW}{\mathscr{W}}
\newcommand*{\scrX}{\mathscr{X}}
\newcommand*{\scrY}{\mathscr{Y}}
\newcommand*{\scrZ}{\mathscr{Z}}

%Section break
\newcommand*{\brk}{
\rule{2in}{.1pt}
}

%General purpose stuff
\DeclareMathOperator{\Aut}{Aut}
\DeclareMathOperator{\ch}{char}
\DeclareMathOperator{\rank}{rank}
\DeclareMathOperator{\End}{End}
\let\Im\relax
\DeclareMathOperator{\Im}{Im}

%Category Theory
\DeclareMathOperator{\Hom}{Hom}
\let\hom\relax
\DeclareMathOperator{\hom}{hom}
\DeclareMathOperator{\id}{id}
\DeclareMathOperator{\coker}{coker}
\DeclareMathOperator{\colim}{colim}
\DeclareMathOperator{\invlim}{\lim_{\leftarrow}}
\DeclareMathOperator{\dirlim}{\lim_{\rightarrow}}

%Commutative Algebra
\DeclareMathOperator{\gldim}{gldim}
\DeclareMathOperator{\projdim}{projdim}
\DeclareMathOperator{\injdim}{injdim}
\DeclareMathOperator{\findim}{findim}
\DeclareMathOperator{\flatdim}{flatdim}
\DeclareMathOperator{\depth}{depth}

%Common Categories
\newcommand*{\modR}{\mathbf{mod}\text{-}R}
\newcommand*{\Rmod}{R\text{-}\mathbf{mod}}
\DeclareMathOperator{\Vectk}{\mathbf{Vect}_k}
\DeclareMathOperator{\Ch}{\mathbf{Ch}}
\newcommand*{\Ab}{\mathbf{Ab}}
\newcommand*{\Grp}{\mathbf{Grp}}
\newcommand*{\Algk}{\mathbf{Alg}_k}
\newcommand*{\Ring}{\mathbf{Ring}}
\newcommand*{\K}{\mathbf{K}}
\newcommand*{\D}{\mathbf{D}}
\newcommand*{\Db}{\mathbf{D}^b}
\newcommand*{\Dpos}{\mathbf{D}^+}
\newcommand*{\Dneg}{\mathbf{D}^-}
\newcommand*{\Dbperf}{\mathbf{D}^b_{\text{perf}}}
\newcommand*{\Dsing}{\mathbf{D}_{sing}}
\DeclareMathOperator{\stmod}{\mathbf{stmod}}
\DeclareMathOperator{\StMod}{\mathbf{StMod}}
\DeclareMathOperator{\sHom}{\underline{Hom}}

%Homological algebra
\DeclareMathOperator{\cone}{cone}
\DeclareMathOperator{\HH}{HH}
\DeclareMathOperator{\Der}{Der}
\DeclareMathOperator{\Ext}{Ext}
\DeclareMathOperator{\Tor}{Tor}

%Lie algebras
\DeclareMathOperator{\ad}{ad}
\newcommand*{\gl}{\mathfrak{gl}}
\let\sl\relax
\newcommand*{\sl}{\mathfrak{sl}}
\let\sp\relax
\newcommand*{\sp}{\mathfrak{sp}}
\newcommand*{\so}{\mathfrak{so}}

% Hacks and Tweaks
% Enumerate will automatically use letters (e.g. part a,b,c,...)
\setenumerate[0]{label=(\alph*)}
% Always use wide tildes
\let\tilde\relax
\newcommand*{\tilde}[1]{\widetilde{#1}}
%raise that Chi!
\DeclareRobustCommand{\Chi}{{\mathpalette\irchi\relax}}
\newcommand{\irchi}[2]{\raisebox{\depth}{$#1\chi$}} 



%Shade definitions
\theoremindent0cm
\theoremheaderfont{\normalfont\bfseries} 
\def\theoremframecommand{\colorbox[rgb]{0.9,1,.8}}
\newshadedtheorem{defn}[thm]{Definition}

%%%%%%%%%%%%%%%%%%%%%%%%%%%%%%%%%%%%%%%%%%%%%%%%%%%%%%%%%%%%%%%%%%%%%%
%%%%%%%%%%%%%%%%%%%%%%% Customize Below %%%%%%%%%%%%%%%%%%%%%%%%%%%%%%
%%%%%%%%%%%%%%%%%%%%%%%%%%%%%%%%%%%%%%%%%%%%%%%%%%%%%%%%%%%%%%%%%%%%%%

%header stuff
\setlength{\headsep}{24pt}  % space between header and text
\pagestyle{fancy}     % set pagestyle for document
\lhead{Notes on the Grassmannian} % put text in header (left side)
\rhead{Nico Courts} % put text in header (right side)
\cfoot{\itshape p. \thepage}
\setlength{\headheight}{15pt}
\allowdisplaybreaks

% Document-Specific Macros
\DeclareMathOperator{\Gr}{Gr}
\DeclareMathOperator{\Flag}{Flag}
\DeclareMathOperator{\GL}{GL}
\newcommand{\Alg}{\mathbf{Alg}}
\newcommand{\CRing}{\mathbf{CRing}}
\DeclareMathOperator{\inv}{inv}
\newcommand{\detq}{\operatorname{det}_q}
\let\k\relax
\newcommand{\k}{\mathbbm{k}}

\begin{document}
%make the title page
\title{Notes on the Grassmannian \vspace{-1ex}}
\author{Nico Courts}
\date{Summer 2019}
\maketitle

\begin{abstract}
	These notes are my summary of the realization of the Grassmanian $\Gr(n,k)$ as a quotient of 
	the Lie group $\GL(n,\k)$. In particular the focus will be on $\k=\bbR$ or $\bbC$. I also talk about quantum 
	deformations of $\GL(n,\k)$ and $\Gr(n,k)$.
\end{abstract}

\section*{Vista: Where We're Headed}
The idea that $\GL_n$ acts on vector subspaces of a Euclidean space should be unsurprising and natural, but 
the upshot to considering this viewpoint is that is allows us to consider this object from the perspective of smooth 
manifold theory. This gives us some great machinery to grasp onto to prove some nice properties 
about $\Gr(n,k)$.

Eventually I will lead us slightly astray by investigating a ``quantum analog'' of $\Gr(n,k)$. This $q$-deformation
is related closely to the kinds of structures I have been thinking about recently: specifically quantum groups. One can think of 
these as ``deformations'' of groups in some controlled way.

\section{The Grassmannian as a Lie group quotient}
In the following, let $\k=\bbR$ or $\bbC$ and let $V=\k^n$. For a great introduction to this theory see \cite{LeeISM},
(specifically chapters 1, 7, and 21), which serves as the primary source for this section.
\subsection{Smooth Manifolds}
Recall that a smooth manifold is a topological manifold equipped with a \textbf{smooth structure}: that is, an atlas comprised 
of smoothly-compatable charts (ones where the transition functions are smooth as maps of Euclidean spaces).
\subsection{The general linear Lie group} 
Let $G=\GL(V)$, initially considered just as a group. In fact, one can apply to $G$ a smooth structure very naturally
by recognizing $\GL(V)\subseteq M_{n\times n}(\k)$, the latter of which is identified with the Euclidean space $\k^{n^2}$,
whose smooth structure is the obvious one (the identity map).

That $G$ itself is a smooth manifold comes from the fact that $\GL(V)=\det^{-1}(\k\setminus\{0\})$, which is 
itself an open set as the inverse image of an open set under a continuous map. So then $G$ is a (codimension zero) open (smooth) submanifold
of $M_{n\times n}(k)$ whose smooth structure is inherited from the ambient space.

In fact, the algebraic and geometric structure of $G$ come together in a very nice way to define what is called 
a \textbf{Lie group:}
\begin{defn}
	Let $G$ be a smooth manifold. Then a \textbf{Lie group} is $G$ along with a group structure given by maps $m:G\times G\to G$ and $i:G\to G$
	(representing multiplication and inverse maps) such that $m$ and $i$ are smooth maps (smooth in charts).
\end{defn}

\begin{prop}
	$G$ is a Lie group.
\end{prop}
\begin{prf}
	Since we have already defined the smooth and group structures on $G$, it suffices to show that 
	$m$ and $i$ (matrix multiplication and the inverse map) are smooth maps.

	Multiplication is smooth since if we consider the $i^{th}$ component function of $m$, where $i=an+b$ for $1\le a,b\le n$, we get (for $\mathbf x$ and $\mathbf y$ in $\k^{n^2}$)
	\[m_i(\mathbf x,\mathbf y)=\sum_{k=1}^{n}\mathbf x_{an+k}\mathbf y_{kn+b}\]
	which is a polynomial in the $x_i$ and $y_i$ and is thus (clearly) smooth.

	That the inverse map is also smooth follows via similar reasoning: since $\det$ is smooth (again a polynomial in the entries),
	computing the adjugate matrix is a smooth operation and so via Cramer's rule (since $\det A\ne 0$)
	\[i(A)=\frac{1}{\det A}\operatorname{adj}A\]
	is also a smooth operation.
\end{prf}

\subsection{The action on \texorpdfstring{$\Gr(n,k)$}{Gr(n,k)}}
The first thing to notice is that there is a natural action 
of $G$ on the set of $k$-dimensional subspaces of $V$ in the following way: let $A\in\GL(V)$ and let $W\le V$ be a $k$-dimensional 
subspace. Then $A(W)$ is also $k$-dimensional since $A$ is invertible\footnote{If you are not convinced, remember that $A$ being invertible means all minors are nonsingular,
so by extending a basis for $W$ to $V$ and looking at $A$ in this basis, the top-left $k\times k$ matrix defines an isomorphism 
of $W$ onto some other subspace of $V$.}.
\begin{prop}
	The action defined above gives a group action of $\GL(V)$ on $\Gr(n,k)$.
\end{prop}
\begin{prf}
	This is clear since the identity matrix $I_n$ fixes subspaces and since matrix multiplication is associative.
\end{prf}

\subsection{Defining a smooth structure on the Grassmanian}
As we are working in the context of smooth manifolds, we need to make some sense of how this action fits into the theory of smooth group actions.
Normally,
\begin{defn}
	If $G$ is a Lie group and $M$ is a smooth manifold, we say $G$ acts smoothly on $M$ if $G$ acts on $M$ as a set and the map 
	\[\rho:G\times M\to M\]
	given by 
	\[\rho(g,m)=g.m\]
	is a smooth map.
\end{defn}
In this case, we don't have a smooth structure (yet!) on $\Gr(n,k)$, so we will define one:

\subsubsection{Stabilizers}
Our first step is to show that the stabilizer of an arbitrary $W\in\Gr(n,k)$, denoted $G_W$, is a closed Lie subgroup (that is a Lie group and closed smooth submanifold). 
To this end:
\begin{thm}[Closed Subgroup Theorem]\label{thm-closed-subgroup}
	Suppose $G$ is a Lie group and $H\subseteq G$ is a subgroup that is also a closed subset of $G$. Then $H$ is an embedded 
	Lie subgroup.
\end{thm}
\begin{rmk}
	The proof of this theorem is two pages of Lie theory and is a little involved for this talk. Jack does 
	a great job (as always) in his proof (cf. \cite[Thm. 20.12]{LeeISM})
\end{rmk}

\noindent Using this theorem we start by proving something about a particular subset:
\begin{prop}\label{prop-GW}
	Fix some basis $\calB$ for $V$. Let $W\le V$ be the subspace of $\k^n$ by setting the last $(n-k)$ coordinates in this basis to zero (which is isomorphic to $\k^k$). Then 
	the stabilizer $G_W$ of $W$ is 
	\[G_W=\left\{\begin{pmatrix}
		A & B\\ 0 & D
	\end{pmatrix}:A\in\GL_k(\k),\ D\in\GL_{n-k}(\k),\ B\in M_{k\times n-k}(\k)\right\}\]
	and this set is a closed Lie subgroup of $\GL(V)$.
\end{prop}
\begin{prf}
	We begin by noticing that $G_W$ as above is precisely the stabilizer of $W$. But because we picked 
	$W$ in the way we did, we know immediately that $\calA\in\GL(V)$ fixes $W$ if the $i^{th}$ coordinate
	of $\calA(\mathbf x)$ is zero for all $\mathbf x=(x_1,\dots,x_k,0,\dots,0)$. Since that $x_i$ are arbitrary,
	we get that $\calA$ is block upper triangular as above. That $A$ and $D$ are invertible comes from the fact that 
	\[\det\calA=\det A\det D\ne 0.\]
	Therefore the stabilizer of $W$ is contained in $G_W$. The reverse inclusion is clear.

	That $G_W$ is a subgroup of $\GL(V)$ then follows from group theory. In light of thm.~\ref{thm-closed-subgroup},
	proving that $G_W$ is a Lie subgroup amounts to showing that $G_W$ forms a closed subset of $\GL(V)$. But this is clear because 
	this is inverse image of all of $\GL_k(\k)$ under $F:\GL(V)\to\GL_k(\k)$ that maps a matrix to the $k\times k$ minor in the upper left corner.

	Equivalently, $G_W$ is defined by the closed condition $\calA_{ij}=0$ if $i\ge k_1$ and $j\le k$. That every such matrix has the form of $G_W$ is 
	clear as argued above.

	Thus $G_W$ is closed, so is a closed Lie subgroup of $G$.
\end{prf}

So we have shown that a particular stabilizer $G_{\k^k}$ is a closed Lie subgroup, but in fact this holds for \textit{any $W\in\Gr(n,k)$:}
\begin{cor}
	For any $W\in\Gr(n,k)$, $G_W$ is a closed Lie subgroup of $G$.
\end{cor}
\begin{prf}
	Simply choose a new basis for $V$ such that all vectors in $W$ are of the form 
	\[(v_1,\dots,v_k,0,\dots,0)\]
	and apply prop.~\ref{prop-GW}.
\end{prf}

\subsubsection{The Quotient Manifold Theorem}
Recall from point-set topology that a proper map is one that pulls back compact sets to compact sets.
\begin{defn}
	An action of a Lie group $G$ on a smooth manifold $M$ is \textbf{proper} if the map 
	\[G\times M\to M\times M\qquad \text{via}\qquad (g,p)\mapsto(g.p,p)\]
	is a proper map.
\end{defn}
\begin{rmk}
	Jack (\cite[Ex. 21.3]{LeeISM}) says that this is a \textit{strictly weaker} condition than saying
	$(g,p)\mapsto g.p$ is proper.
\end{rmk}
\begin{thm}[Quotient Manifold Theorem]\label{thm-QMT}
	Let $G$ be a Lie group acting \textit{smoothly, freely, and properly} on a smooth manifold $M$. Then 
	the orbit space $M/G$ is a topological manifold of dimension $\dim M-\dim G$ and has a unique smooth structure 
	making the quotient map $\pi:M\to M/G$ a smooth submersion.
\end{thm}
\begin{prf}
	As before, see Jack's proof \cite[thm. 21.10]{LeeISM}.
\end{prf}
\begin{rmk}
	In case this wasn't obvious, this is a rather high-powered result and is the primary facilitator of what follows.
\end{rmk}

\subsubsection{Homogeneous Spaces}
\begin{defn}
	A smooth manifold $M$ endowed with a \textit{transitive, smooth} action by a Lie group $G$ is called
	a \textbf{homogeneous space.}
\end{defn}
Then we use the quotient manifold theorem to construct a homogeneous space for ourselves (following part of the proof of \cite[thm. 21.17]{LeeISM}):
\begin{prop}
	Let $W$ be an arbitrary element in $\Gr(n,k)$. Then if $G=\GL(V)$, $G/G_W$ is a homogeneous space given by the action
	\[g\cdot (g'G_W)=(gg')G_W\]
\end{prop}
\begin{prf}
	Consider the action of $G_W$ on $G$ by left multiplication. This action is free since if $hg=g$, we get $h=e$. The action is a restriction of 
	$G$ on itself by left multiplication, which is smooth by virtue of being a Lie group, so the action is smooth.

	To see the action is proper, we use the fact (proved in \cite[prop. 21.5(a,b)]{LeeISM}) that the action is proper if and only if $(h_i)$ is a sequence in $G_W$ and 
	$(g_i)$ are a sequence in $G$, if both $(h_i)$ and $(h_ig_i)$ converge then the $(g_i)$ converge.

	Let $(g_i)$ be a convergent sequence in $G$ and $(h_i)$ a sequence in $G_W$ such that $(h_ig_i)$ converges. Then by the continuity of 
	the $G$ action on itself, $h_i=(h_ig_i)g_i^{-1}$ converges to a point in $G$. Since $G_W$ is closed in $G$, this limit converges in $G_W$.

	Thus we can apply thm~\ref{thm-QMT} to say that $G/G_W$ has a unique smooth manifold structure making the quotient map $\pi:G\to G/G_W$ a smooth submersion. The action 
	\[g\cdot g'G_W=(gg')G_W\]
	is transitive and smooth, giving us our homogeneous space.
\end{prf}
\subsubsection{Bringing it home}
Now that we have that $G/G_W$ is a smooth manifold, consider the map
\[F:G/G_W\to \Gr(n,k)\]
be defined by 
\[F(A\cdot G_W)=A(W).\]

\begin{lem}
	$F$ defined as above is a $G$-equivariant bijection.
\end{lem}
\begin{prf}
	This is a well-defined map since if $gG_W=g'G_W$, we get that $g=g'h$ for some $h\in G_W$, thus
	\[F(gG_W)=g\cdot W=g'hW=g'W=F(g'G_W).\]

	Then that this map is surjective follows since the action of $G$ on $\Gr(n,k)$ has a single orbit. That 
	the map is injective follows since if the images of the same, this means precisely that $gW=g'W$, so $g^{-1}g'\in G_W$,
	proving $gG_W=g'G_W$.

	$G$-equivariance is a simple consequence of the fact 
	\[g\cdot F(g'G_W)=g\cdot (g'W)=(gg')W=F(gg'G_W)=F(g(g'G_W)).\]
\end{prf}

\noindent Now we can give $\Gr(n,k)$ a smooth structure such that $F$ is a diffeomorphism\footnote{Open sets are precisely the images of open sets of $G/G_V$ and the smooth 
structure is the one pulled back through $F$.}. Notice now that the action of $G$ on $\Gr(n,k)$ is smooth since we can write 
\[g\cdot V=F(g\cdot F^{-1}(V))\]
since $F$ and the action of $G$ on $G/G_W$ is smooth.

We still need to establish that such a smooth structure is, in some sense, \textit{the} structure that makes 
the action provided into a smooth action. This will be the smooth structure we give the Grassmannian.

We need one more theorem from smooth manifolds to finish this off:
\begin{thm}[Equivariant Rank Theorem]
	Let $M$ and $N$ be smooth manifolds and $G$ be a Lie group. If $F:M\to N$ is a smooth, bijective, $G$-equivariant map with respect to 
	a smooth, transitive action of $G$ on $M$ and any smooth action of $G$ on $N$, then $F$ is a diffeomorphism.
\end{thm}
\begin{prf}
	See \cite[thm 7.25]{LeeISM}.
\end{prf}
\begin{lem}
	There is only one smooth structure on $X=\Gr(n,k)$ such that the action of $G$ on $X$ is smooth.
\end{lem}
\begin{prf}
	Let $\tilde X$ be $X$ with any other smooth structure. Then if $G$ acts smoothly on $\tilde X$, the (smooth) action $\theta^{(W)}:G\to \tilde X$ given by $\theta^{(W)}(g)=g\cdot W$ 
	descends to the quotient to give us a smooth map $\calF:G/G_W\to \tilde X$, which as before can be shown to be a $G$-equivariant bijection.

	But then by the equivariant rank theorem since the action of $G$ on $G/G_W$ is smooth and transitive, and the action of $G$ on $\tilde X$ is smooth,
	the map $\calF$ is a diffeomorphism. But then in particular since $F$ gave us a diffeomorphism from $G/G_W$ to $X$, 
	we have a diffeomorphism $X\cong \tilde X$, so the smooth structure on $X$ is unique.
\end{prf}
\subsection{Denouement: \texorpdfstring{$\Gr(n,k)$}{Gr(n,k)} is compact.}
I found a neat proof of this online \href{https://math.stackexchange.com/questions/148088/compactness-of-the-grassmannian}{here}.
\begin{prop}
	$\Gr(n,k)$ is compact.
\end{prop}
\begin{prf}
	Let $S=\{v\in V=\k^n: \|v\|=1\}$. Notice that $S\cong \bbS^{n-1}$, the $n-1$ sphere in $\k^n$, which is compact. Then $S^d=S\times \cdots\times S$ is compact as well.

	Considering $S^d$ as some choice of $d$ unit vectors in $\k^n$, we can consider the map 
	\[S^d\to\k^{\binom{d}{2}}\qquad\text{via}\qquad (v_1\dots,v_d)\mapsto (v_1\cdot v_2,v_1\cdot v_3,\dots,v_1\cdot v_n,v_2\cdot v_3,\dots,v_{n-1}\cdot v_n)\]
	and notice that pulling back $\mathbf 0$ in this map gets us a (closed whence) compact subspace $C$ of $S^d$ corresponding to the 
	sets of $d$ orthonormal vectors in $\k^n$.

	Finally define the equivalence relation on $C$ where $c_1\sim c_2$ if $\operatorname{span} c_1=\operatorname{span} c_2$. Then the quotient 
	$C/\sim$ is precisely $\Gr(n,k)$ which, as a quotient of a compact space, is compact.
\end{prf}

\section{Aside: Quantized Grassmanians}
In this section I will take a little time to develop some of the things I have been learning about quickly,
specifically developing non-commutative algebraic geometry by relying on intuitions from traditional algebraic geometry 
while ``deforming'' things in a controlled way so that we can use our intuition to develop analogous ideas.

\subsection{Algebra vs. Geometry}
The discussion that follows is a relatively constructive method for $q$-deforming the Grassmannian and studying the action it 
inherits from $\GL_q(n)$. It gives a good flavor for the field, but is very strongly rooted in algebra.

That is, the methods we describe take the coordinate ring of the Grassmannian and deform it purely in the realm of algebra, with the understanding 
that there is some parallel geometric structure that is being analogously twisted.

Late in my research, I found a paper by Aaron Lauve \cite{lauve} takes the results of Taft and Towber and reimagines them in a context that 
preserves a certain level of geometry. I haven't had time to read it in its entirety but it may be a good place to look for those interested in the geometry.

\subsection{Hopf Algebras}
The leading idea here is that Hopf algebras lie at the heart of a lot of modern algebra---especially representation theory. Sometimes you are working with 
a Hopf algebra without knowing it!

Let's begin with a definition:
\begin{defn}
	A \textbf{Hopf algebra} is a $k$-vector space $A$ equipped with maps 
	\begin{gather*}
		\mu:A\otimes A\to A\\
		\eta:k\to A\\
		\Delta:A\to A\otimes A\\
		\varepsilon:A\to k\\
		S:A\to A
	\end{gather*}
	such that $\mu$ and $\eta$ form an associative algebra structure on $A$, $\Delta$ and $\varepsilon$ form a coassociative coalgebra structure on $A$,
	$\mu$ is a coalgebra morphism, $\Delta$ is an algebra morphism (so far we have defined a bialgebra) and furthermore $S:A\to A$ is an antihomomorphism:
	\[S(ab)=S(b)S(a)\]
	and such that all the maps satisfy the following:
	\[\mu\circ S\otimes\id\circ \Delta(a)=S(a_1)a_2=a_2S(a_2)=\mu\circ \id\otimes S\circ\Delta(x)=\eta\circ\varepsilon(x)\]
\end{defn}
\begin{rmk}
	We used Sweedler notation above to simplify things although it looks deceptively simple to the uninitiated.
\end{rmk}
\begin{rmk}
	In shorter terms, 
\end{rmk}

\subsubsection{Examples of Hopf algebras}
These algebras pop up relatively frequently! For instance, the tensor, symmetric, and exterior algebras on a vector space are all 
readily made into Hopf algebras.

I am a representation theorist, so my primary examples tend to revolve around 
the group algebra $k[G]$ of a group. It is a Hopf algebra where you define the Hopf algebra structure by making all 
elements of $G$ grouplike\footnote{$\Delta(x)=x\otimes x$, $\varepsilon(x)=1$ and $S(x)=x^{-1}$} and extending $k$-linearly.

You can also dualize things somewhat and consider the algebra of regular functions on $G$, denoted $kG$ or $k(G)$. When $G$ is finite,
these are isomorphic but when $G$ is infinite these can be quite different. In either of these contexts one can interchange ``group''
with ``group scheme'' or ``algebraic group''. A quite nice treatment of the affine case (that is when the Hopf algebra is finite dimensional over a field)
can be found in \cite{waterhouse}. If you are interested in a highly representation-theoretic treatment of algebraic groups, you can can check out \cite{jantzen-algebraic}
or \cite{Milne}.

Finally the last way these arise in nature is in the universal enveloping algebra $\calU(\frakg)$ of a Lie algebra $\frakg$. One can find the theory here developed in Jantzen's 
other book \cite{jantzen-quantum}.

\subsection{Quantum deformations of Hopf algebras}
In all of the above examples, the Hopf algebra is either commutative or cocommutative and in fact you get some nice equivalences of categories 
between, for instance, group schemes over a field $k$ and commutative Hopf algebras over $k$. The ``quantum perspective'' is the tendency to 
ask the question: \textit{what about the non-commutative, non-cocommutative Hopf algebras?}

In some way, the study of quantum groups is the attempt to consider the Hopf algebras that fall in the gaps between the more ``natural'' ones.

\subsection{The idea}
The primary source for this section is Taft and Towber's \textit{Quantum deformation of flag schemes and Grassmann schemes} \cite{quantum-flag}.
Our tact here is to build up an algebra (thought of as the ring of functions on the flag or Grassmannian scheme) and through deformation of this algebra,
$q$-deform the space.

\begin{rmk}
	Generally when working with vector spaces one can more-or-less forget about which basis is chosen when working in coordinates. Unfortunately here
	the commutativity (or lack thereof) of a quantum deformation is highly dependent on the choice of \textit{ordering} of any basis. Thus we will be considering commutative rings $\k$ and 
	$\k$-algebras freely generated by some fixed \textbf{ordered} basis $\calB=\{v_1,\dots,v_n\}$. 

	In what follows I will suppress the basis $\calB$ (e.g. write $\Flag^n(\calB)$ as $\Flag^n$) but remember that it is vital to the construction.
\end{rmk}

\subsection{The affine flag scheme \texorpdfstring{$\Flag^n(\calB)$}{Flag-n(B)}}
\begin{rmk}
	I will admit that here is the most difficult part for me: defining these objects\footnote{As well as seeing why these objects are what their names imply they should be} requires some geometry 
	that is beyond my understanding at the moment. I will do my best, but the geometers among you may notice some 
	holes in my justification. Corrections are warmly welcomed.
\end{rmk}
The idea here is to define the scheme $\Flag^n:=\Flag^n(\calB):\CRing\to\Set$ which assigns to any commutative ring $\k$ the (commutative, associative, unital) $\k$-algebra
$\Flag^n(\k)$ generated over $\k$ by the elements of $\Lambda^\ast(\calB)$
\[v_{i_1}\wedge\cdots\wedge v_{i_r}\qquad (1\le r\le n,\ \forall j,\  1\le i_j\le n)\]
subject to the relations: $v_{i_1}\wedge\cdots\wedge v_{i_s}=0$ if two $i_k$ coincide and if $\sigma\in\frakS_n$,
\[v_{i_{\sigma 1}}\wedge\cdots\wedge v_{i_{\sigma s}}=(\operatorname{sgn}\sigma)\  v_{i_1}\wedge\cdots\wedge v_{i_s}\]
together with the relation 
\[\sum_{1\le \lambda_1\le\cdots\le\lambda_r\le t+r}(-1)^{\sum \lambda_i}\bigwedge_{k\ne \lambda_i \forall i}v_{k}\times_{\Lambda^+(\calB)}(v_{\lambda_1}\wedge\cdots\wedge v_{\lambda_r}\wedge v_{j_1}\wedge\cdots\wedge v_{j_{s-r}})=0\]
for all $1\le r\le s\le t\le n$ and all choices of $i_1,\dots,i_{t+r}$ and $j_1,\dots,j_{s-r}$.

\subsubsection{The grading}
This algebra admits a $P(n)$ grading where $P(n)$ is the semigroup (here no inverses) of nonincreasing sequences of positive integers
\[(\alpha_1,\dots,\alpha_k)\]
with $n\ge \alpha_1\ge\cdots\ge \alpha_1\ge 1$ where multiplication is defined by (disjoint) union and sorting the multiset. The $P(n)$ grading determines the ``\textbf{shape}'' of a monomial in 
$\Flag^n(\k)$ in the following way: we can write each monomial $\omega\in\Flag^n(\k)$ as $\omega_1\cdots\omega_r$ where $\omega_i\in\Lambda^{\alpha_i}(\calB)$
and such that $n\ge\alpha_1\ge\cdots\ge\alpha_r\ge 1$. For notation, if $\alpha=(\alpha_1,\dots,\alpha_r)$, write 
\[\Flag^\alpha(\k)=\Lambda^\alpha(\calB,\k)=\Lambda^\alpha(\calB)\]
to be the $\k$-span of all products of shape $\alpha$.

This gives us a graded algebra structure (called the \textbf{shape algebra})
\[\Lambda^+(\calB)=\bigoplus_{\alpha\in P(n)}\Lambda^\alpha(\calB)\]
of $\Flag^n(\k)$.

The upshot here is that each $\Lambda^\alpha$ inherits a natural $\GL(n,\k)$ action by acting on each $v_i$ independently.

Then defining a deformation of this algebra amounts to deforming $\GL(n,\k)$.

\subsection{The affine Grassmann scheme \texorpdfstring{$\Gr^{t,n}(\calB)$}{Gr(t,n)(B)}}
We define the \textbf{affine Grassmann scheme} $\Gr^{t,n}:=\Gr^{t,n}(\calB)$ to be a closed subscheme of $\Flag^n$ such that 
$\Gr^{t,n}(\k)$ is the $\k$-subalgebra of $\Flag^n(\k)$ generated by the $t$-fold wedges in $\calB$. Notice this also inherits a grading, 
but this time we can view it as a $\bbZ$-grading since all monomials in $\Gr^{t,n}(\k)$ are of shape $(t,t,\dots,t)$.

\subsection{Quantized \texorpdfstring{$\GL(n)$}{GL(n)}}
Let $q$ be an indeterminate (often thought of as an element in the base ring [e.g an odd root of unity] and always as a \textit{deformation parameter})
and let $\k_q:=\k[q,q^{-1}]$, the Laurent polynomials in $q$. To proceed we need the \textit{(strict) inversion statistic} seen in the combinatorics of $\frakS_n$.
\begin{defn}
	If $\sigma\in\frakS_n$, let 
	\[\inv\sigma=\#\{(i,j)|1\le i\le j\le n,\ \sigma(i)>\sigma(j)\}\]
\end{defn}

Then we can go on to define $M_q(n,\k)$ to be the $\k_q$-algebra generated by $n^2$ elements $X_i^j$ for $1\le i,j\le n$
subject to the $q$-commutativity relations
\begin{equation} 
\begin{gathered}
	X_{k}^jX_i^j=qX_i^jX_{k}^j,\qquad \text{if } i<k\\
	X_i^kX_i^j=qX_i^jX_i^k,\qquad\text{if } j<k\\
	X_k^lX_i^j=X_i^jX_k^l+(q-q^{-1})X_i^lX_k^j,\qquad\text{if } i<k \text{ and } j<l\\
	X_k^lX_i^j=X_i^jX_k^l\qquad\text{otherwise.}
\end{gathered} \label{eqn:qcomm}
\end{equation}

We define the \textbf{$q$-determinant} $\det_q$ to be (thinking of $\mathbf X=(X_i^j)$ as a matrix)
\[\detq(\mathbf X)=\detq=\sum_{\sigma\in\frakS_n}(-q)^{-\inv\sigma} X_1^{\sigma 1}\cdots X_n^{\sigma n}\]
and can therefore define the $q$-analog of $\GL(n,\k)$ (written $\GL_q(n,\k)$) to be the $\k_q$-algebra generated by 
the $X_i^j$ and an indeterminate $D$ satisfying the relations in (1) above as well as 
\[D\cdot\detq =\detq\cdot D=1\in\k\]
as well as $DX_i^j=X_i^jD$ for all $i$ and $j$.

The coalgebra structure given by $\Delta(X_i^j)=\sum_k X_i^k\otimes X_k^j$, $\Delta(D)=D\otimes D$,
$\varepsilon(X_i^j)=\delta_i^j$ and $\varepsilon(D)=1$.

The antipode $S$ is given as 
\[S(X_i^j)=(-q)^{j-i}\ |X^{1\cdots\hat i\cdots n}_{1\cdots\hat j\cdots n}|_q\ D\]
where in general $|X^{i_1\cdots i_s}_{j_1\cdots j_s}|_q$ denotes the $q$ determinant of the $s\times s$ matrix formed by the rows $j$ and the columns $i$ of $\mathbf X$ in that order.
Finally $S(D)=\detq$. This gives us a Hopf algebra structure on $\GL_q(n,\k)$ and thus realizes $\GL_q(n)$ as a quantum group.

Notice that we can give $\k$ a $\k_q$-module structure by the evaluation map $ev:\k_q\to\k$ sending $q$ to 1. Under this structure,
we get 
\[\GL_q(n,\k)\otimes_{\k_q}\k\cong \k[\GL(n,\k)]\]
so indeed this quantization collapses to the regular $\k$ functions on $\GL(n)$, indicating we have found the (perhaps \textit{a}) ``correct'' object.

\subsection{The quantized Grassmannian}
Then we can define $\Gr_q^{t,n}$ to be the $\k_q$ algebra generated by the $t$-fold wedges (notice that we are losing the intuition of the wedge product now so this is just a notational designation)
in $\Lambda_q^+(\calB)$,
the quantized shape algebra where we replace the usual relations with
\[v_{i_{\sigma 1}}\wedge\cdots\wedge v_{i_{\sigma s}}=(-q)^{\inv\sigma}\  v_{i_1}\wedge\cdots\wedge v_{i_s}\]
as well as the quantum Young symmetry relation
\[\sum_{1\le \lambda_1\le\cdots\le\lambda_r\le t+r}(-q)^{-\inv(i_1,\dots,i_{\hat\lambda_i}\cdots i_{t+r},i_{\lambda_1},\dots,i_{\lambda_r})}\bigwedge_{k\ne \lambda_i \forall i}v_{k}\times_{\Lambda_q^+(\calB)}(v_{\lambda_1}\wedge\cdots\wedge v_{\lambda_r}\wedge v_{j_1}\wedge\cdots\wedge v_{j_{s-r}})=0\]
along with a (rather nasty) $q$-commutativity relation (i.e. $\Lambda_q^+$ is no longer commutative)
\begin{align*}
	&(v_{j_1}\wedge\cdots\wedge v_{j_r})\cdot (v_{i_1}\wedge\cdots\wedge v_{i_r})=\\
	&\sum_{1\le \lambda_1\le\cdots\le\lambda_r\le s}(-q)^{\inv(\lambda_1,\dots,\lambda_r,1,\dots,\hat\lambda_i,\dots,s)}(v_{j_1}\wedge\cdots\wedge v_{j_r}\wedge v_{i_1}\wedge\cdots\wedge\hat v_{i_{\lambda_i}}\wedge\cdots v_{i_s})(v_{\lambda_1}\wedge\cdots\wedge v_{\lambda_r})
\end{align*}

The punchline (after many pages of theorems and lemmas) is that $\Gr_q^{t,n}(\k)$ inherits a natural $\GL_q(n)$-module structure as well as a $\GL_q(n)$-comodule structure 
from the structure map 
\[\rho:\Lambda_q^t(\calB)\to \GL_q(n,\k)\otimes \Lambda_q^t(\calB)\]
where 
\[\rho(v_{\lambda_1}\wedge\cdots\wedge v_{\lambda_t})=\sum_{1\le \mu_1\le\cdots\le \mu_t\le n}|X^{\mu_1,\dots,\mu_t}_{\lambda_1,\dots,\lambda_t}|_q\otimes(v_{\mu_1}\wedge\cdots\wedge v_{\mu_t})\]

Furthermore, this comodule structure is the unique comodule structure extending the natural comodule structure of $\GL_q(n)$ on itself (cf. the ``normal representation'').

Finally the good stuff: let $L_q^t(\calB)$ denote the (unital) $\k_q$ subalgebra of $M_q(n,\k)$ generated 
by the quantum minors $|X^{1,\dots,t}_{\lambda_1,\dots,\lambda_t}|_q$ for $1\le \lambda_i\le n$. There is an isomorphism of $\GL_q(n,\k)$-comodule algebras 
between $\Lambda^t_q(\calB)$ and $L^t_q(\calB)$ by sending every such element to $v_{\lambda_1}\wedge\cdots\wedge v_{\lambda_t}$. So this gives us an easy generating set 
for the quantum Grassmannian over $\k_q$.

Of course it is also shown that the map $ev:\k_q\to \k$ sending $q$ to 1 induces an isomorphism 
\[\Lambda_q^t(\calB)\otimes_{\k_q}\k\cong \Lambda^t(\calB)\cong \k[\Gr^{t,n}]\]
and since we used the same map $ev$, we get that this isomorphism is compatible with the $\GL_q(n,\k)$ action.
%%%%%%%%%%%%%%%%%%%%%%%%%%%%%%%%%%%%%%
%%%%%%%%%%  Bibliography %%%%%%%%%%%%%
%%%%%%%%%%%%%%%%%%%%%%%%%%%%%%%%%%%%%%
\medskip

\printbibliography

\end{document}